
A questo punto della relazione,  abbiamo compreso come funziona la crittografia classica e come i computer quantisci sono in grado di violarla, ora possiamo quindi entrare nel secondo punto fondamentale di questo elaborato : la crittografia post-quantistica (PQC).
Prima però servirà un ultimo preambolo, visto che abbiamo compreso che l'interra crittografia si basa su problemi matematici, analizziamo in breve quali sono le classi di complessità computazionali dei problemi matematici che andremo poi ad analizzare per capirne la sicurezza. 

\subsection*{Introduzione}

\paragraph{Classi di complessità e problemi matematici}

Per comprendere perché la crittografia post-quantistica rappresenta una soluzione efficace, introduciamo la gerarchia delle classi di complessità computazionale per capire dove si collocano i diversi problemi che ci saranno utili in seguito. 

\textbf{Il problema P vs NP}
Uno dei problemmi ancora irrisolti nella matematica e informatica teorica è appunto il problema P vs NP, difatti è inserito tra i sette problemi del millennio dal \footnote{\href{https://www.claymath.org/millennium/p-vs-np/}{Clay Mathematic Institute}}. 
La questione riguarda la differenza tra il "risolvere" un problema e il "verificare" una soluzione giò data. 

\begin{itemize}
    \item \textbf{P (Polynomial time)}: Problemi che possono essere risolti in tempo polinomiale da un algoritmo deterministico. Questi problemi sono considerati "facili" da risolvere. Esempio: ordinamento di una lista.
    \item \textbf{NP (Nondeterministic Polynomial time)}: Problemi per i quali, data una possibile soluzione, è possibile verificare la correttezza in tempo polinomiale. Esempio: il problema del cammino hamiltoniano.
    \item \textbf{Np-hard}: Problemi almeno difficili quanto i problemi più difficili in NP. Formalmente, un problema NP-hard è tale se può essere ridotto in tempo polinomiale a un qualisiasi problema in NP, percui risolvere un problema NP-hard significa dimmostrare che $P = NP$. Esempio: il problema del commesso viaggiatore (TSP).
\end{itemize}

Da qui emerge una domanda cruciale: per ogni problema la cui soluzione è facile da verficare (NP) è anche facile trovare una solzuione (P)? Se $P = NP$, allora ogni volta che possiamo controllare rapidamente una soluzione deve esistere anche un modo veloce per trovarla. La comunità scientifica concorda che $P \neq NP$, il che implica che esistono problemi in intrinecamente difficili per i quali trovare la soluzione richiede tempi esponenziali, anche se verificarli è immediato, e questo rappresenta il fondamento della sicurezza crittografica.

\paragraph{BQP}
Nel caso della nostra analisi, è importante introdurre in questa gerarchia anche la classe BQP (Bounded-error Quantum Polynomial time), che rappresenta l'insieme dei problemi risolvibili efficientemente da un computer quantistico. 
Attualmmente, sebbene non sia dimostrato, la comunità scientifica ritiene che BQP non contenga NP-hard. In altre parole, si ipotizza che nemmeno un computer quantistico possa risolvere in modo efficiente problemi come il TSP. Questo è fondamentale perché implica che esistono problemi matematici che rimangono difficili da risolvere anche rispetto al calcolo quantistico e che quindi possono essere utilizzati come base per la crittografia post-quantistica.
In questa classe di problemi rientrano infatti RSA e ECC, che come abbiamo visto sono vulnerabili per via della loro struttura matematica basata sulla periodicità.

\paragraph{PQC}
La crittografia post-quantistica sposta quindi la sua sicurezza dai problemmi di classe NP-Intermediate (di cui fanno parte RSA ed ECC) a nuove famiglie matematiche che, allo stato attuale della ricerca, non presentano vulnerabilità esponenziali quantistiche, in particolare motli di questi sono legati a problemi NP-hard. 

\subsection{Lattice-baces}

Crittografia basata sui reticoli (Lattice-based)\footnote{\href{https://preprints.org/manuscript/202508.0555/v1}{Quantum Computing and Cryptography}}

È considerata la famiglia più promettente grazie a un ottimo equilibrio tra sicurezza e prestazioni. Questi algoritmi si basano sulla difficoltà di problemi geometrici nei reticoli, come il Shortest Vector Problem (SVP).

\begin{itemize}
\item CRYSTALS-Kyber (ML-KEM): Standardizzato dal NIST per lo scambio di chiavi (Key Encapsulation Mechanism), offre chiavi di dimensioni ridotte (\textasciitilde1–2 KB).
\item CRYSTALS-Dilithium (ML-DSA): Standardizzato per le firme digitali.
\item Falcon (FN-DSA): Un altro schema di firma basato su reticoli, noto per la sua compattezza.
\item NTRU: Uno dei sistemi basati su reticoli più longevi e studiati.
\end{itemize}
\newpage

\subsection{Code-based}

Crittografia basata sui codici (Code-based)\footnote{\href{https://en.wikipedia.org/wiki/Post-quantum_cryptography}{Wikipedia: Code-based cryptography}}

Si basa sulla difficoltà di decodificare un codice lineare casuale.

\begin{itemize}
\item Classic McEliece: È l'algoritmo più antico (proposto nel 1978) e ha resistito alla crittanalisi classica e quantistica per oltre 40 anni. Lo svantaggio principale risiede nelle dimensioni delle chiavi pubbliche estremamente grandi (spesso nell'ordine dei megabyte).
\item HQC: Recentemente selezionato dal NIST per la futura standardizzazione.
\end{itemize}
\newpage

\subsection{Hash-based}

Crittografia basata su Hash (Hash-based)\footnote{\href{https://csrc.nist.gov/pubs/fips/205/final}{NIST FIPS 205: Stateless Hash-Based Digital Signature Standard}}

Questi algoritmi creano firme digitali basandosi esclusivamente sulla sicurezza delle funzioni hash crittografiche.

\begin{itemize}
\item SPHINCS+ (SLH-DSA): È una firma digitale standardizzata che non richiede assunzioni matematiche complesse se non la resistenza alle collisioni dell'hash scelto. Le firme sono più grandi rispetto ad altri schemi (\textasciitilde40 KB), ma la sicurezza è ritenuta molto solida.
\item XMSS / LMS: Schemi di firma "stateful", adatti per scenari specifici come gli aggiornamenti del firmware.
\end{itemize}
\newpage

\subsection{Multivariate}

Crittografia multivariata (Multivariate)\footnote{\href{https://en.wikipedia.org/wiki/Post-quantum_cryptography}{Wikipedia: Multivariate and Isogeny-based cryptography}}

Si basa sulla difficoltà di risolvere sistemi di equazioni polinomiali multivariate, che è un problema NP-difficile.

\begin{itemize}
\item Rainbow: Uno schema di firma che offre firme molto piccole e processi di firma rapidi. Tuttavia, la sua sicurezza è stata messa in discussione da recenti attacchi algebrici.
\end{itemize}
\newpage

\subsection{Isogeny-based} 

Crittografia basata sulle isogenie (Isogeny-based)

Utilizza le proprietà delle mappe (isogenie) tra curve ellittiche supersingolari.

\begin{itemize}
\item CSIDH / SIDH: Offrono le chiavi più piccole tra tutti i candidati PQC, ma i tempi di calcolo sono generalmente più lenti. Nota: lo schema SIDH/SIKE è stato violato nel 2022 da un attacco classico, sebbene altre costruzioni basate su isogenie rimangano valide.
\end{itemize}
\newpage

\subsection{Crittografia simmetrica} 

Crittografia simmetrica e resistenza quantistica\footnote{\href{https://preprints.org/manuscript/202508.0555/v1}{Grover’s Algorithm and Symmetric Key Lengths}}

Per quanto riguarda i sistemi a chiave simmetrica (come l'AES) e le funzioni hash, come abbiamo accennato nel capitolo 4, sono già intrinsecamente resistenti agli attacchi quantistici.
Infatti abbiamo brevemente introddo l'algoritmo di Grover che fornisce un'accelerazione quadratica per le ricerche brute-force ma non rappresenta veramente una grossa minaccia come shor. Pertanto per mantenere un livello di sicurezza di 128 bit, è sufficiente raddoppiare la lunghezza della chiave (utilizzando AES-256 anziché AES-128).

\paragraph{Considerazioni sulla migrazione}

Per mitigare i rischi, molte aziende (come Apple con PQ3 o Google) stanno adottando un approccio di crittografia ibrida, combinando un algoritmo classico con uno post-quantistico per garantire sicurezza anche nel caso in cui uno dei due si rivelasse vulnerabile in futuro.