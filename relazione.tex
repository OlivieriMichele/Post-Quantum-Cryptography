\documentclass[a4paper,12pt]{article}
\usepackage[utf8]{inputenc}
\usepackage{amsmath}
\usepackage{graphicx}
\usepackage{hyperref}

\title{Crittografia Post-Quantum}
\author{Olivieri Michele}
\date{\today}

\begin{document}

\maketitle

\newpage

\tableofcontents

\newpage

\section{Introduzione}

1.1 Contesto e motivazioni

Per parlare di crittografia post-quantistica, è fondamentale comprendere il contesto in cui essa si inserisce.  
Come visto a lezione la crittografia classica pone le sue fondamneta su problemi computazionalmente difficili
per i quali non esistono algoritmi efficienti in grado di risolverli in tempi polinomiali.

In questo contesto entrano in gioco i computer quantistici, dispositivi che sfruttano i principi della 
meccanica quantistica per eseguire calcoli in modo radicalmente diverso rispetto ai computer classici
che quindi li rende capaci di risolvere quei problemi matematici ritenuti difficili o intrattabili per i computer convenzionali. 

Tuttavia se fino ad un decennio fa sembravano non essere una minaccia, il rapido sviluppo del calcolo quantistico 
ha destato crescente preoccupazione riguardo alla sicurezza dei sistemi crittografici attualmente in uso. 

Sebbene i computer quantistici siano ancora in una fase sperimentale, i loro potenziali avanzamenti rappresentano una minaccia 
concreta per gli algoritmi crittografici classici su cui si basa gran parte della sicurezza informatica moderna. 
In particolare, algoritmi come:
RSA (Rivest-Shamir-Adleman), ECC (Elliptic Curve Cryptography) e DSA (Digital Signature Algorithm), 
potrebbero essere facilmente compromessi dall'uso di potenti computer quantistici grazie all'algoritmo di Shor
che è infatti capace di risolvere in tempi polinomiali il probelamm della fattorizzazione e del logarimto discreto
su cui di basano gli algoritmi citati.

Questa vulnerabilità mette a rischio non solo la riservatezza delle comunicazioni attuali, ma anche l'integrità e l'autenticità 
dei dati, ponendo una seria minaccia a lungo termine per la sicurezza delle infrastrutture digitali globali. 

Molti scienziati ritengono che la costruzione di computer quantistici su larga scala sia ormai solo una sfida ingegneristica, 
con alcuni ingegneri che prevedono il loro sviluppo entro i prossimi vent'anni. Considerando che storicamente ci sono voluti 
quasi due decenni per implementare l'attuale infrastruttura crittografica, è necessario iniziare ora a preparare sistemi 
in grado di resistere al calcolo quantistico.

L'obiettivo della crittografia post-quantistica è quindii sviluppare algoritmi crittografici sicuri sia contro i computer quantistici 
che classici, garantendo così un futuro sicuro anche in un'era dominata dai computer quantistici.

\bibliographystyle{plain}
\bibliography{riferimenti}

\section{Fondamenti della crittografia classica e delle minacce quantistiche}

\section{Principali algoritmi della crittografia quantistica}

\section{Protocolli post-quantum}

\section{Sfide e considerazioni pratiche}

\section{Stato attuale delle ricerche e delle implementazioni}

\section{Applicazioni e scenari futuri}

\section{Conclusioni}

\end{document}